%!TEX TS-program = xelatex
%!TEX encoding = UTF-8 Unicode
% Awesome CV LaTeX Template for CV/Resume
%
% This template has been downloaded from:
% https://github.com/posquit0/Awesome-CV
%
% Author:
% Claud D. Park <posquit0.bj@gmail.com>
% http://www.posquit0.com
%
% Template license:
% CC BY-SA 4.0 (https://creativecommons.org/licenses/by-sa/4.0/)
%

%-------------------------------------------------------------------------------
% CONFIGURATIONS
%-------------------------------------------------------------------------------
% A4 paper size by default, use 'letterpaper' for US letter
\documentclass[10pt, a4paper]{awesome-cv}
\usepackage{fontspec}
\usepackage{fontawesome}
\usepackage{xltxtra}

% Configure page margins with geometry
\geometry{left=1.4cm, top=.8cm, right=1.4cm, bottom=1.8cm, footskip=.5cm}

% Specify the location of the included fonts
\fontdir[fonts/]

% Color for highlights
% Awesome Colors: awesome-emerald, awesome-skyblue, awesome-red, awesome-pink, awesome-orange
%                 awesome-nephritis, awesome-concrete, awesome-darknight
\colorlet{awesome}{awesome-red}
% Uncomment if you would like to specify your own color
% \definecolor{awesome}{HTML}{3E6D9C}

% Colors for text
% Uncomment if you would like to specify your own color
\definecolor{darktext}{HTML}{414141}
% \definecolor{text}{HTML}{333333}
% \definecolor{graytext}{HTML}{5D5D5D}
% \definecolor{lighttext}{HTML}{999999}
% \definecolor{sectiondivider}{HTML}{5D5D5D}

% Set false if you don't want to highlight section with awesome color
\setbool{acvSectionColorHighlight}{true}

% If you would like to change the social information separator from a pipe (|) to something else
\renewcommand{\acvHeaderSocialSep}{\quad\textbar\quad}


%-------------------------------------------------------------------------------
%	PERSONAL INFORMATION
%	Comment any of the lines below if they are not required
%-------------------------------------------------------------------------------
% Available options: circle|rectangle,edge/noedge,left/right
\photo[left]{../../../images/grum.png}
%\name{Romain}{Gallet}
\name{Grum}{Ltd}
\position{Software Architecture{\enskip\cdotp\enskip}Data engineering}
%\address{42-8, Bangbae-ro 15-gil, Seocho-gu, Seoul, 00681, Rep. of KOREA}

\mobile{(+44) 787 243 6423 }
\email{info@grumlimited.co.uk}
\homepage{grumlimited.co.uk}
\github{grumlimited}
\linkedin{romaingallet}
% \gitlab{gitlab-id}
% \stackoverflow{SO-id}{SO-name}
% \twitter{@twit}
% \skype{skype-id}
% \reddit{reddit-id}
% \medium{madium-id}
% \kaggle{kaggle-id}
% \hackerrank{hackerrank-id}
% \googlescholar{googlescholar-id}{name-to-display}
%% \firstname and \lastname will be used
% \googlescholar{googlescholar-id}{}
% \extrainfo{extra information}

\quote{\textcolor{awesome}{\textbf{CONTRACTS ONLY - LONDON ONLY}}\\
\vspace{1em}
Engineering and consultancy services within the media, financial and publishing industries.\\
\textbf{Grum Ltd} provides expertise in high availability architectures around streaming data pipelines.}

\hypersetup{%
  pdftitle={Grum Limited~~~·~~~Consultancy},
  pdfauthor={Romain Gallet},
  pdfsubject={Grum Limited~~~·~~~Consultancy},
  pdfkeywords={CV, Grum, Consultancy}
}

%-------------------------------------------------------------------------------
\begin{document}

% Print the header with above personal information
% Give optional argument to change alignment(C: center, L: left, R: right)
\makecvheader[C]

% Print the footer with 3 arguments(<left>, <center>, <right>)
% Leave any of these blank if they are not needed
\makecvfooter
  %{\today}
  {}
  {Grum Limited~~~·~~~Consultancy}
  {\thepage}


%-------------------------------------------------------------------------------
%	CV/RESUME CONTENT
%	Each section is imported separately, open each file in turn to modify content
%-------------------------------------------------------------------------------
%%-------------------------------------------------------------------------------
%	SECTION TITLE
%-------------------------------------------------------------------------------

\cvsection{Summary}
%\vspace{2ex}


%-------------------------------------------------------------------------------
%	CONTENT
%-------------------------------------------------------------------------------
\begin{cvparagraph}

%---------------------------------------------------------

Dynamic software architect and product strategist with a history of guiding successful projects spanning various domains (including data architecture, social networks, mobile apps, games, robotics, and AI), team sizes ranging from 2 to 10 members, and diverse settings, from agile B2C environments to R\&D projects backed by €3.5 million in public funding.

With expertise at the intersection of engineering and product leadership, I excel in delivering cutting-edge solutions that drive business growth and enhance user experiences \see{n42r.github.io}.

\end{cvparagraph}

%-------------------------------------------------------------------------------
%	SECTION TITLE
%-------------------------------------------------------------------------------
\cvsection{Experience}


%-------------------------------------------------------------------------------
%	CONTENT
%-------------------------------------------------------------------------------
\begin{cventries}

%---------------------------------------------------------
  \cventry
    {Senior Technical Lead} % Job title
    {Bellroy (TrikeApps was merged into Bellroy in Apr. 2018)} % Organization
    {Melbourne, Australia} % Location
    {Nov. 2017 - Present} % Date(s)
    {
      \begin{cvitems} % Description(s) of tasks/responsibilities
        \item {Actively participating in architectural discussion with Senior Devs to clarify and document architectural goals.}
        \item {Guide juniors on individual stories to provide big picture and ensure code quality.}
        \item {Regularly perform code reviews to encourage best practices.}
        \item {Lead EDI (Electronic Data Interchange) integration efforts for multiple high-profile clients.}
        \item {Built ETL pipelines for syncing orders from digital marketplaces. Wrote a parser to generate rails migrations from API spec for core business entities of interest to optimize the time spent to build future ETL pipelines.}
        \item {Supported the business with technical requirements for a business entity restructure.}
        \item {Wrote a caching layer for a throttled external API which has saved > 5M API calls (171 days worth of API calls), which has saved systems from being blocked countless times.}
        \item {Wrote a Background job throttler which can automatically filter low value background jobs and defer them to next day (while it was running low on external API calls for the day) to be retried automatically so critical business processes could continue running. This saved a lot of tech support time.}
        \item {Worked with a colleague to plan and move multiple self-hosted postgres databases into AWS aurora without much downtime. Performed mental simulations and upfront analysis to cut down disruptions of business processes.}
        \item {Implemented a generalized shipping alert rules which allowed Logistcs to flag shipments that matched certain criterias to enforce better business processes.}
        \item {Participated in and heavily contributed to a massive refactoring of a order fulfillment pipeline, following clean architecture practices, which led to the system processing 200K+ background jobs without a single failure on a recent sale spike in 24 hours.}
        \item {In an effort to build a CI/CD pipeline on AWS, migrated CI from Travis to AWS CodeBuild. Used AWS CloudFormation to build the stack (including Docker images). It also cut down cost by ~70\% while also cutting down spec run time for our largest project by several minutes. Also Built an Elm component for internal dashboard to show CodeBuild build statues.}
        \item {Spent personal time to build a demo CD pipeline from scratch on EKS using first Terraform , then cloudformation with Blue/Green deploys which contributed to the approval of CI/CD project.}
        \item {Paired with another dev to build a staging Kubernetes Cluster on EKS using cloudformation, following good networking and security practices to progress on the CI/CD project.}
        \item {Spent personal time to build Grafana dashboard and add hubot (with slack integration) to encourage DevOps/chatops practices.}
        \item {Leading data migration effort of an undergoing ERP integration project.}
      \end{cvitems}
    }
%---------------------------------------------------------
  \cventry
    {Team Lead} % Job title
    {TrikeApps} % Organization
    {Melbourne, Australia} % Location
    {Apr. 2017 - Oct. 2017} % Date(s)
    {
      \begin{cvitems} % Description(s) of tasks/responsibilities
        \item {Contributed to creating an internal DSL for specifying shipping rules in primary client's e-commerce store which allowed specifying complex shipping rules.}
        \item {Prepared internal systems to be able to work with (with the accounting system integration) a business entity change for primary client.}
        \item {Managed 2 other Junior devs, participated in sprint planning and estimation and organized retrospectives.}
        \item {Took the responsibility of meeting with Transactions team weekly, and worked closely with them to maintain and support the rapid growth of the business.}
        \item {Integrated multiple new digital marketplaces, including automated order fulfillment, settlement/payment report processing etc.}
      \end{cvitems}
    }
%---------------------------------------------------------
  \cventry
    {Ruby Developer} % Job title
    {TrikeApps} % Organization
    {Melbourne, Australia} % Location
    {Apr. 2016 - Apr. 2017} % Date(s)
    {
      \begin{cvitems} % Description(s) of tasks/responsibilities
        \item {Upgraded multiple internal apps to Rails 5.}
        \item {Added Apple Pay Payment method on primary client's e-commerce store which significantly increased conversion from qualifying Safari users.}
        \item {Built a S3 backed drag-n-drop product image management system for primary client's e-commerce store.}
        \item {Converted a legacy ruby app into a Rails one for consistency.}
        \item {Wrote a data/business validity checker which let others write SQL queries to run periodically and report them on business critical processes without help from tech support.}
        \item {Built and maintained a self-hosted analytics platform (Snowplow) for the data analytics team, involved learning about, setting up and maintaining the full stack including AWS CloudFront, Hadoop/Spark jobs, AWS Athena, AWS Redshift. This supported high performance, low overhead batch event processing for all systems.}
      \end{cvitems}
    }
%---------------------------------------------------------
  \cventry
    {PHP Software Developer} % Job title
    {Astute Payroll} % Organization
    {Melbourne, Australia} % Location
    {Feb. 2016 - Mar. 2017} % Date(s)
    {
      \begin{cvitems} % Description(s) of tasks/responsibilities
        \item {Refactored the document management system from legacy code to S3 backed clean component using Twig templates.}
        \item {Refactored payslip rendering service to a domain service which was then used in cli, API and user facing features.}
        \item {Wrote a robust regression tester script in PHP which could generate thousands of payslips using old and new renderer and report on differences to build confidence before cutover.}
        \item {Worked on third party integrations.}
      \end{cvitems}
    }

%---------------------------------------------------------
  \cventry
    {Web Developer} % Job title
    {Adgate Media} % Organization
    {Remote} % Location
    {Feb. 2015 - Jan. 2016} % Date(s)
    {
      \begin{cvitems} % Description(s) of tasks/responsibilities
        \item {Implemented complex ideas into working solutions using Laravel.}
        \item {Used OOD best practices to produce testable functional code.}
        \item {Learnt about Unit and functional testing and started following TDD.}
        \item {Built a multi-room chat functionality with ReactPHP on the backend and ReactJS on the frontend with Websocket and ZeroMQ (with support for custom commands).}
        \item {Added real-time frontend update support via websocket, to deliver critical information without page refresh.}
        \item {Built cache decorators with repository patterns to make caching layer transparent.}
        \item {Used AWS SQS for background job processing.}
        \item {Built a permission system which allowed feature restriction on user level.}
      \end{cvitems}
    }

%---------------------------------------------------------
  \cventry
    {Software Engineer} % Job title
    {Noobis Inc} % Organization
    {Dhaka, Bangladesh} % Location
    {Jul. 2011 - Sep. 2015} % Date(s)
    {
      \begin{cvitems} % Description(s) of tasks/responsibilities
        \item {Analysed client specifications and project requirements, and turned them into functional solutions.}
        \item {Prepared regular progress reports.}
        \item {Updating and Maintaining a large Magento installation including operational maintenance (learnt a lot about linux server administration).}
        \item {Developed in-house apps with Laravel and AngularJS.}
        \item {Identified and communicated changing priorities and business objectives to other team members and distributed workload.}
        \item {Built MVPs for investor presentations.}
      \end{cvitems}
    }
%---------------------------------------------------------
  \cventry
    {Web Developer} % Job title
    {Allmoxy Inc} % Organization
    {Remote Part Time} % Location
    {July. 2012 - May. 2013} % Date(s)
    {
      \begin{cvitems} % Description(s) of tasks/responsibilities
        \item {Added various reports into the existing app which gave more insight into the business.}
        \item {Extended the custom MVC the client was using with more features.}
        \item {Extended the frontend built with PrototypeJS with more features.}
        \item {Created a documentation framework which let the customer support add video walkthrough and in turn saved time on support ticket resolution.}
      \end{cvitems}
    }

%---------------------------------------------------------
  \cventry
    {Web Developer Intern} % Job title
    {Leevio Inc.} % Organization
    {Remote Part Time} % Location
    {Dec. 2009 - Jun. 2011 } % Date(s)
    {
      \begin{cvitems} % Description(s) of tasks/responsibilities
      \item {Learnt about VCS (subversion).}
      \item {Contributed to the development of inhouse and client apps.}
      \item {Built few mobile apps with Titanium.}
      \item {Learnt and used Zend and CodeIgniter PHP frameworks.}
      \end{cvitems}
    }
%---------------------------------------------------------
\end{cventries}

%-------------------------------------------------------------------------------
%	SECTION TITLE
%-------------------------------------------------------------------------------
\cvsection{Education}


%-------------------------------------------------------------------------------
%	CONTENT
%-------------------------------------------------------------------------------
\begin{cventries}

%---------------------------------------------------------
  \cventry
    {Master of Science (M.Sc), Computer Science} % Degree
    {University of Central Florida} % Institution
    {Orlando, FL} % Location
    {Aug. 2008 - May. 2011} % Date(s)
    {
    }

  \cventry
    {Bachelor of Science (B.Sc.), Computer Science} % Degree
    {} % Institution
    {} % Location
    {Aug. 2001 - May. 2006} % Date(s)
    {
	\begin{cvitems}
		\item {Recipient of Florida Bright Futures Scholarship}
	\end{cvitems}
    }

%---------------------------------------------------------
\end{cventries}

%-------------------------------------------------------------------------------
%	SECTION TITLE
%-------------------------------------------------------------------------------
\cvsection{Honors \& Awards}


%-------------------------------------------------------------------------------
%	SUBSECTION TITLE
%-------------------------------------------------------------------------------
\cvsubsection{International}


%-------------------------------------------------------------------------------
%	CONTENT
%-------------------------------------------------------------------------------
\begin{cvhonors}

%---------------------------------------------------------
  \cvhonor
    {Finalist} % Award
    {DEFCON 25th CTF Hacking Competition World Final} % Event
    {Las Vegas, U.S.A} % Location
    {2017} % Date(s)

%---------------------------------------------------------
  \cvhonor
    {Finalist} % Award
    {DEFCON 22nd CTF Hacking Competition World Final} % Event
    {Las Vegas, U.S.A} % Location
    {2014} % Date(s)

%---------------------------------------------------------
  \cvhonor
    {Finalist} % Award
    {DEFCON 21st CTF Hacking Competition World Final} % Event
    {Las Vegas, U.S.A} % Location
    {2013} % Date(s)

%---------------------------------------------------------
  \cvhonor
    {Finalist} % Award
    {DEFCON 19th CTF Hacking Competition World Final} % Event
    {Las Vegas, U.S.A} % Location
    {2011} % Date(s)

%---------------------------------------------------------
  \cvhonor
    {6th Place} % Award
    {SECUINSIDE Hacking Competition World Final} % Event
    {Seoul, S.Korea} % Location
    {2012} % Date(s)

%---------------------------------------------------------
\end{cvhonors}


%-------------------------------------------------------------------------------
%	SUBSECTION TITLE
%-------------------------------------------------------------------------------
\cvsubsection{Domestic}


%-------------------------------------------------------------------------------
%	CONTENT
%-------------------------------------------------------------------------------
\begin{cvhonors}

%---------------------------------------------------------
  \cvhonor
    {3rd Place} % Award
    {WITHCON Hacking Competition Final} % Event
    {Seoul, S.Korea} % Location
    {2015} % Date(s)

%---------------------------------------------------------
  \cvhonor
    {Silver Prize} % Award
    {KISA HDCON Hacking Competition Final} % Event
    {Seoul, S.Korea} % Location
    {2013} % Date(s)

%---------------------------------------------------------
\end{cvhonors}

%%-------------------------------------------------------------------------------
%	SECTION TITLE
%-------------------------------------------------------------------------------
\cvsection{Presentation}


%-------------------------------------------------------------------------------
%	CONTENT
%-------------------------------------------------------------------------------
\begin{cventries}

%---------------------------------------------------------
  \cventry
    {Presenter for <DEFCON 20th : The way to go to Las Vegas>} % Role
    {6th CodeEngn (Reverse Engineering Conference)} % Event
    {Seoul, S.Korea} % Location
    {Jul. 2012} % Date(s)
    {
      \begin{cvitems} % Description(s)
        \item {Introduced CTF(Capture the Flag) hacking competition and advanced techniques and strategy for CTF}
      \end{cvitems}
    }

%---------------------------------------------------------
\end{cventries}

%-------------------------------------------------------------------------------
%	SECTION TITLE
%-------------------------------------------------------------------------------
\cvsection{Writing}


%-------------------------------------------------------------------------------
%	CONTENT
%-------------------------------------------------------------------------------
\begin{cventries}

%---------------------------------------------------------
  \cventry
    {Founder \& Writer} % Role
    {A Guide for Developers in Start-up} % Title
    {Facebook Page} % Location
    {Jan. 2015 - PRESENT} % Date(s)
    {
      \begin{cvitems} % Description(s)
        \item {Drafted daily news for developers in Korea about IT technologies, issues about start-up.}
      \end{cvitems}
    }

%---------------------------------------------------------
  \cventry
    {Undergraduate Student Reporter} % Role
    {AhnLab} % Title
    {S.Korea} % Location
    {Oct. 2012 - Jul. 2013} % Date(s)
    {
      \begin{cvitems} % Description(s)
        \item {Drafted reports about IT trends and Security issues on AhnLab Company magazine.}
      \end{cvitems}
    }

%---------------------------------------------------------
\end{cventries}

%%-------------------------------------------------------------------------------
%	SECTION TITLE
%-------------------------------------------------------------------------------
\cvsection{Program Committees}


%-------------------------------------------------------------------------------
%	CONTENT
%-------------------------------------------------------------------------------
\begin{cvhonors}

%---------------------------------------------------------
  \cvhonor
    {Problem Writer} % Position
    {2016 CODEGATE Hacking Competition World Final} % Committee
    {S.Korea} % Location
    {2016} % Date(s)

%---------------------------------------------------------
  \cvhonor
    {Organizer \& Co-director} % Position
    {1st POSTECH Hackathon} % Committee
    {S.Korea} % Location
    {2013} % Date(s)

%---------------------------------------------------------
\end{cvhonors}

%%-------------------------------------------------------------------------------
%	SECTION TITLE
%-------------------------------------------------------------------------------
\cvsection{Extracurricular Activity}


%-------------------------------------------------------------------------------
%	CONTENT
%-------------------------------------------------------------------------------
\begin{cventries}

%---------------------------------------------------------
  \cventry
    {Core Member \& President at 2013} % Affiliation/role
    {PoApper (Developers' Network of POSTECH)} % Organization/group
    {Pohang, S.Korea} % Location
    {Jun. 2010 - Jun. 2017} % Date(s)
    {
      \begin{cvitems} % Description(s) of experience/contributions/knowledge
        \item {Reformed the society focusing on software engineering and building network on and off campus.}
        \item {Proposed various marketing and network activities to raise awareness.}
      \end{cvitems}
    }

%---------------------------------------------------------
  \cventry
    {Member} % Affiliation/role
    {PLUS (Laboratory for UNIX Security in POSTECH)} % Organization/group
    {Pohang, S.Korea} % Location
    {Sep. 2010 - Oct. 2011} % Date(s)
    {
      \begin{cvitems} % Description(s) of experience/contributions/knowledge
        \item {Gained expertise in hacking \& security areas, especially about internal of operating system based on UNIX and several exploit techniques.}
        \item {Participated on several hacking competition and won a good award.}
        \item {Conducted periodic security checks on overall IT system as a member of POSTECH CERT.}
        \item {Conducted penetration testing commissioned by national agency and corporation.}
      \end{cvitems}
    }

%---------------------------------------------------------
\end{cventries}



%-------------------------------------------------------------------------------
\end{document}
