%!TEX TS-program = xelatex
%!TEX encoding = UTF-8 Unicode
% Awesome CV LaTeX Template for Cover Letter
%
% This template has been downloaded from:
% https://github.com/posquit0/Awesome-CV
%
% Authors:
% Claud D. Park <posquit0.bj@gmail.com>
% Lars Richter <mail@ayeks.de>
%
% Template license:
% CC BY-SA 4.0 (https://creativecommons.org/licenses/by-sa/4.0/)
%


%-------------------------------------------------------------------------------
% CONFIGURATIONS
%-------------------------------------------------------------------------------
% A4 paper size by default, use 'letterpaper' for US letter
\documentclass[11pt, a4paper]{awesome-cv}

% Configure page margins with geometry
\geometry{left=1.4cm, top=.8cm, right=1.4cm, bottom=1.8cm, footskip=.5cm}

% Specify the location of the included fonts
\fontdir[fonts/]

% Color for highlights
% Awesome Colors: awesome-emerald, awesome-skyblue, awesome-red, awesome-pink, awesome-orange
%                 awesome-nephritis, awesome-concrete, awesome-darknight
\colorlet{awesome}{awesome-red}
% Uncomment if you would like to specify your own color
% \definecolor{awesome}{HTML}{CA63A8}

% Colors for text
% Uncomment if you would like to specify your own color
% \definecolor{darktext}{HTML}{414141}
% \definecolor{text}{HTML}{333333}
% \definecolor{graytext}{HTML}{5D5D5D}
% \definecolor{lighttext}{HTML}{999999}

% Set false if you don't want to highlight section with awesome color
\setbool{acvSectionColorHighlight}{true}

% If you would like to change the social information separator from a pipe (|) to something else
\renewcommand{\acvHeaderSocialSep}{\quad\textbar\quad}


%-------------------------------------------------------------------------------
%	PERSONAL INFORMATION
%	Comment any of the lines below if they are not required
%-------------------------------------------------------------------------------
% Available options: circle|rectangle,edge/noedge,left/right
% \photo[circle,noedge,left]{./examples/profile}
\name{Jonas}{Falck}
\position{Bioinfornatician - Data Analyst}
\address{Muzenplaats 4, 6525JA, Nijmegen, Netherlands}

\mobile{(+31) 655707314}
\email{jonas@falcken.de}
%\homepage{www.posquit0.com}
\github{joe-nas}
%\linkedin{posquit0}
% \gitlab{gitlab-id}
% \stackoverflow{SO-id}{SO-name}
% \twitter{@twit}
% \skype{skype-id}
% \reddit{reddit-id}
% \medium{madium-id}
% \googlescholar{googlescholar-id}{name-to-display}
%% \firstname and \lastname will be used
% \googlescholar{googlescholar-id}{}
% \extrainfo{extra informations}

% \quote{``rtfm"}


%-------------------------------------------------------------------------------
%	LETTER INFORMATION
%	All of the below lines must be filled out
%-------------------------------------------------------------------------------
% The company being applied to
\recipient
  {Steven Castelein}
  {Radboud University Medical Center – Genomediagnostics}
% The date on the letter, default is the date of compilation
\letterdate{\today}
% The title of the letter
\lettertitle{Application: Genomediagnostics Bioinformatician at Radboud University Medical Center}
% How the letter is opened
\letteropening{Dear drs. Steven Castelein}
% How the letter is closed
\letterclosing{Sincerely,}
% Any enclosures with the letter
%\letterenclosure[Attached]{Curriculum Vitae}


%-------------------------------------------------------------------------------
\begin{document}

% Print the header with above personal informations
% Give optional argument to change alignment(C: center, L: left, R: right)
\makecvheader[R]

% Print the footer with 3 arguments(<left>, <center>, <right>)
% Leave any of these blank if they are not needed
\makecvfooter
  {\today}
  {Jonas Falck~~~·~~~Motivation}
  {}

% Print the title with above letter informations
\makelettertitle

%-------------------------------------------------------------------------------
%	LETTER CONTENT
%-------------------------------------------------------------------------------
\begin{cvletter}

\lettersection{About Me}
After completing my civil service duty, I moved from Germany to the Netherlands to study medical biology in Nijmegen. 
I completed my bachelor’s degree and began my master's in medical biology at the Radboud University. 
During this time, both in- and outside of university, I focused on developing IT skills such as setting up and managing headless Linux systems, programming with a focus on bioinformatics, machine learning and dynamic websites. 
I honed my bioinformatic skills during my master's internships in the Radboud UMC human genetics department, as well as in the comparative genomics group. 
Due to financial and personal reasons, I started working in logistics and was unable to finish my master’s degree. 
However, I am determined to pursue my passion for human biology and IT and work in a field that allows me to utilize my potential while having a greater societal impact.
Apart from my professional goals, I enjoy going to concerts, bouldering, and most importantly, spending time with my friends and family.
\lettersection{Why Genomediagnostics at Radboud UMC:}
Genomediagnostics combines the fields that I am passionate about - human biology, (sequencing-)technology. 
In Nijmegen, the town I call my home, the Radboud UMC provides cutting-edge diagnostics, research, and patient care and I would be proud to be involved in that process. 
As a trained biologist and a family member of a cancer patient, I understand the importance of making informed decisions based on the best available data.
In the genomediagnostics department, data meeting the highest standards is generated and I could utilize my skills to ensure a smooth production environment.
Building on my experience in bioinformatics and with Linux servers, I could get hands-on experience with a large compute cluster and workflow management systems.
Moreover, I could support co-workers who may have less knowledge of bioinformatics or difficulty interpreting the data we generate. 
I value collaborating with international and interdisciplinary teams to exchange knowledge and provide support to each others.
Based on my internships, I know that this is possible at Radboud UMC. 
Therefore, I would like to join your team at the genome diagnostics department.
\lettersection{Why Me?}
While my career path may not have been straightforward, I have gained a lot of experience along the way, making me the ideal candidate for this position. 
I am not only highly motivated to work in diagnostics with big sequencing-data and big computers, but I also possess the necessary technical skills for working on server infrastructure, with large amounts of data and analyzing complex omics data.
As such, I have experience in setting up and working with headless Linux systems and containerized applications, as well as programming/scripting in R, Python, bash and the command line.
I have used numerous Bioconductor packages and command-line tools commonly used in bioinformatic analysis. 
Developing my own R package (see CV), working on hobby projects and supporting friends and peers on technical issues have given me ample troubleshooting experience. 
Furthermore, I have experience in communicating bioinformatic methods and results to researchers, regardless of their background in bioinformatics.
I am intrinsically motivated to learn and take pride in my autodidacticism, which has enabled me to acquire most of the skills presented in my CV. 
While I value being able to work independently, I also appreciate having knowledgeable peers around to discuss stubborn problems. 
As a hands-on and down-to-earth person, as well as a hard worker, I would be pleased to meet with you in March.
\end{cvletter}
%-------------------------------------------------------------------------------
% Print the signature and enclosures with above letter informations
\makeletterclosing

\end{document}
