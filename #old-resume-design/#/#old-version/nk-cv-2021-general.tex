%%%%%%%%%%%%%%%%%%%%%%%%%%%%%%%%%%%%%%%%%%%%%%%%%%%%%%%%%%%%%%%%%%%%%%%%
%%%%%%%%%%%%%%%%%%%%%% Simple LaTeX CV Template %%%%%%%%%%%%%%%%%%%%%%%%
%%%%%%%%%%%%%%%%%%%%%%%%%%%%%%%%%%%%%%%%%%%%%%%%%%%%%%%%%%%%%%%%%%%%%%%%

%%%%%%%%%%%%%%%%%%%%%%%%%%%%%%%%%%%%%%%%%%%%%%%%%%%%%%%%%%%%%%%%%%%%%%%%
%% NOTE: If you find that it says                                     %%
%%                                                                    %%
%%                           1 of ??                                  %%
%%                                                                    %%
%% at the bottom of your first page, this means that the AUX file     %%
%% was not available when you ran LaTeX on this source. Simply RERUN  %%
%% LaTeX to get the ``??'' replaced with the number of the last page  %%
%% of the document. The AUX file will be generated on the first run   %%
%% of LaTeX and used on the secondrun to fill in all of the          %%
%% references.                                                         %%
%%%%%%%%%%%%%%%%%%%%%%%%%%%%%%%%%%%%%%%%%%%%%%%%%%%%%%%%%%%%%%%%%%%%%%%%

%%%%%%%%%%%%%%%%%%%%%%%%%%%% Document Setup %%%%%%%%%%%%%%%%%%%%%%%%%%%%

% Don't like 10pt? Try 11pt or 12pt
\documentclass[10pt]{article}

% The automated optical recognition software used to digitize resume
% information works best with fonts that do not have serifs. This
% command uses a sans serif font throughout. Uncomment both lines (or at
% least the second) to restore a Roman font (i.e., a font with serifs).
%\usepackage{times}
%\renewcommand{\familydefault}{\sfdefault}

% This is a helpful package that puts math inside length specifications
\usepackage{calc}

\usepackage{fancyhdr}


% Layout: Puts the section titles on left side of page
\reversemarginpar

%
%         PAPER SIZE, PAGE NUMBER, AND DOCUMENT LAYOUT NOTES:
%
% The next \usepackage line changes the layout for CV style section
% headings as marginal notes. It also sets up the paper size as either
% letter or A4. By default, letter was used. If A4 paper is desired,
% comment out the letterpaper lines and uncomment the a4paper lines.
%
% As you can see, the margin widths and section title widths can be
% easily adjusted.
%
% ALSO: Notice that the includefoot option can be commented OUT in order
% to put the PAGE NUMBER *IN* the bottom margin. This will make the
% effective text area larger.
%
% IF YOU WISH TO REMOVE THE ``of LASTPAGE'' next to each page number,
% see the note about the +LP and -LP lines below. Comment out the +LP
% and uncomment the -LP.
%
% IF YOU WISH TO REMOVE PAGE NUMBERS, be sure that the includefoot line
% is uncommented and ALSO uncomment the \pagestyle{empty} a few lines
% below.
%

%% Use these lines for letter-sized paper
\usepackage[paper=letterpaper,
           % includefoot, % Uncomment to put page number above margin
            marginparwidth=1.2in,     % Length of section titles
            marginparsep=.05in,       % Space between titles and text
            margin=1in,               % 1 inch margins
            includemp,
	 bottom = 2.5cm]{geometry}
%% Use these lines for A4-sized paper
%\usepackage[paper=a4paper,
%            %includefoot, % Uncomment to put page number above margin
%            marginparwidth=30.5mm,    % Length of section titles
%            marginparsep=1.5mm,       % Space between titles and text
%            margin=25mm,              % 25mm margins
%            includemp]{geometry}

%% More layout: Get rid of indenting throughout entire document
\setlength{\parindent}{0in}

\usepackage[shortlabels]{enumitem}

% Simpler bibsections for CV sections
% (thanks to natbib for inspiration)
%
% * For lists of references with hanging indents and no numbers:
%
%   \begin{bibsection}
%       \item ...
%   \end{bibsection}
%
% * For numbered lists of references (with hanging indents):
%
%   \begin{bibenum}
%       \item ...
%   \end{bibenum}
%
%   Note that bibenum numbers continuously throughout. To reset the
%   counter, use
%
%   \restartlist{bibenum}
%
%   at the place where you want the numbering to reset.

\makeatletter
\newlength{\bibhang}
\setlength{\bibhang}{1em}
\newlength{\bibsep}
 {\@listi \global\bibsep\itemsep \global\advance\bibsep by\parsep}
\newlist{bibsection}{itemize}{3}
\setlist[bibsection]{label=,leftmargin=\bibhang,%
        itemindent=-\bibhang,
        itemsep=\bibsep,parsep=\z@,partopsep=0pt,
        topsep=0pt}
\newlist{bibenum}{enumerate}{3}
\setlist[bibenum]{label=[\arabic*],resume,leftmargin={\bibhang+\widthof{[999]}},%
        itemindent=-\bibhang,
        itemsep=\bibsep,parsep=\z@,partopsep=0pt,
        topsep=0pt}
\let\oldendbibenum\endbibenum
\def\endbibenum{\oldendbibenum\vspace{-.6\baselineskip}}
\let\oldendbibsection\endbibsection
\def\endbibsection{\oldendbibsection\vspace{-.6\baselineskip}}
\makeatother

%% Reference the last page in the page number
%
% NOTE: comment the +LP line and uncomment the -LP line to have page
%       numbers without the ``of ##'' last page reference)
%
% NOTE: uncomment the \pagestyle{empty} line to get rid of all page
%       numbers (make sure includefoot is commented out above)
%
\usepackage{fancyhdr,lastpage}
\pagestyle{fancy}
%\pagestyle{empty}      % Uncomment this to get rid of page numbers
\fancyhf{}\renewcommand{\headrulewidth}{0pt}
\fancyfootoffset{\marginparsep+\marginparwidth}
\newlength{\footpageshift}
\setlength{\footpageshift}
          {0.5\textwidth+0.5\marginparsep+0.5\marginparwidth-2in}
\lfoot{\hspace{\footpageshift}%
       \parbox{4in}{\, \hfill %
                    \arabic{page} of \protect\pageref*{LastPage} % +LP
%                    \arabic{page}                               % -LP
                    \hfill \,}}

% Finally, give us PDF bookmarks

\usepackage[usenames,dvipsnames]{xcolor}

\usepackage{color,hyperref}
\definecolor{darkblue}{rgb}{0.0,0.0,0.3}
\hypersetup{colorlinks,breaklinks,
            linkcolor=Blue,urlcolor=Blue,
            anchorcolor=Blue,citecolor=Blue}

%%%%%%%%%%%%%%%%%%%%%%%% End Document Setup %%%%%%%%%%%%%%%%%%%%%%%%%%%%


%%%%%%%%%%%%%%%%%%%%%%%%%%% Helper Commands %%%%%%%%%%%%%%%%%%%%%%%%%%%%

%%% HEADING AT TOP OF CURRICULUM VITAE

% The title (name) with a horizontal rule under it
% (optional argument typesets an object right-justified across from name
%  as well)
%
% Usage: \makeheading{name}
%        OR
%        \makeheading[right_object]{name}
%
% Place at top of document. It should be the first thing.
% If ``right_object'' is provided in the square-braced optional
% argument, it will be right justified on the same line as ``name'' at
% the top of the CV. For example:
%
%       \makeheading[\emph{Curriculum vitae}]{Your Name}
%
% will put an emphasized ``Curriculum vitae'' at the top of the document
% as a title. Likewise, a picture could be included:
%
%   \makeheading[\includegraphics[height=1.5in]{my_picutre}]{Your Name}
%
% the picture will be flush right across from the name.
\newcommand{\makeheading}[2][]%
        {\hspace*{-\marginparsep minus \marginparwidth}%
         \begin{minipage}[t]{\textwidth+\marginparwidth+\marginparsep}%
             {\large \bfseries #2 \hfill #1}\\[-0.15\baselineskip]%
                 \rule{\columnwidth}{1pt}%
         \end{minipage}}

%%% SECTION HEADINGS

% The section headings. Flush left in small caps down pseudo-margin.
%
% Usage: \section{section name}
\renewcommand{\section}[1]{\pagebreak[3]%
    \vspace{1.3\baselineskip}%
    \phantomsection\addcontentsline{toc}{section}{#1}%
    \noindent\llap{\scshape\smash{\parbox[t]{\marginparwidth}{\hyphenpenalty=10000\raggedright #1}}}%
    \vspace{-\baselineskip}\par}

%%% LISTS

% This macro alters a list by removing some of the space that follows the list
% (is used by lists below)
\newcommand*\fixendlist[1]{%
    \expandafter\let\csname preFixEndListend#1\expandafter\endcsname\csname end#1\endcsname
    \expandafter\def\csname end#1\endcsname{\csname preFixEndListend#1\endcsname\vspace{-0.6\baselineskip}}}

% These macros help ensure that items in outer-type lists do not get
% separated from the next line by a page break
% (they are used by lists below)
\let\originalItem\item
\newcommand*\fixouterlist[1]{%
    \expandafter\let\csname preFixOuterList#1\expandafter\endcsname\csname #1\endcsname
    \expandafter\def\csname #1\endcsname{\csname preFixOuterList#1\endcsname\let\oldItem\item\def\item{\pagebreak[2]\oldItem}}
    \expandafter\let\csname preFixOuterListend#1\expandafter\endcsname\csname end#1\endcsname
    \expandafter\def\csname end#1\endcsname{\let\item\oldItem\csname preFixOuterListend#1\endcsname}}
\newcommand*\fixinnerlist[1]{%
    \expandafter\let\csname preFixInnerList#1\expandafter\endcsname\csname #1\endcsname
    \expandafter\def\csname #1\endcsname{\let\oldItem\item\let\item\originalItem\csname preFixInnerList#1\endcsname}
    \expandafter\let\csname preFixInnerListend#1\expandafter\endcsname\csname end#1\endcsname
    \expandafter\def\csname end#1\endcsname{\csname preFixInnerListend#1\endcsname\let\item\oldItem}}

% An itemize-style list with lots of space between items
%
% Usage:
%   \begin{outerlist}
%       \item ...    % (or \item[] for no bullet)
%   \end{outerlist}
\newlist{outerlist}{itemize}{3}
    \setlist[outerlist]{label=\enskip\textbullet,leftmargin=*}
    \fixendlist{outerlist}
    \fixouterlist{outerlist}

% An environment IDENTICAL to outerlist that has better pre-list spacing
% when used as the first thing in a \section
%
% Usage:
%   \begin{lonelist}
%       \item ...    % (or \item[] for no bullet)
%   \end{lonelist}
\newlist{lonelist}{itemize}{3}
    \setlist[lonelist]{label=\enskip\textbullet,leftmargin=*,partopsep=0pt,topsep=0pt}
    \fixendlist{lonelist}
    \fixouterlist{lonelist}

% An itemize-style list with little space between items
%
% Usage:
%   \begin{innerlist}
%       \item ...    % (or \item[] for no bullet)
%   \end{innerlist}
\newlist{innerlist}{itemize}{3}
    \setlist[innerlist]{label=\enskip\textbullet,leftmargin=*,parsep=0pt,itemsep=0pt,topsep=0pt,partopsep=0pt}
    \fixinnerlist{innerlist}

% An environment IDENTICAL to innerlist that has better pre-list spacing
% when used as the first thing in a \section
%
% Usage:
%   \begin{loneinnerlist}
%       \item ...    % (or \item[] for no bullet)
%   \end{loneinnerlist}
\newlist{loneinnerlist}{itemize}{3}
    \setlist[loneinnerlist]{label=\enskip\textbullet,leftmargin=*,parsep=0pt,itemsep=0pt,topsep=0pt,partopsep=0pt}
    \fixendlist{loneinnerlist}
    \fixinnerlist{loneinnerlist}

%%% EXTRA SPACE

% To add some paragraph space between lines.
% This also tells LaTeX to preferably break a page on one of these gaps
% if there is a needed pagebreak nearby.
\newcommand{\blankline}{\quad\pagebreak[3]}
\newcommand{\halfblankline}{\quad\vspace{-0.5\baselineskip}\pagebreak[3]}

%%% FORMATTING MACROS

% Uses hyperref to link DOI
\newcommand\doilink[1]{\href{http://dx.doi.org/#1}{#1}}
\newcommand\doi[1]{doi:\doilink{#1}}

% For \url{SOME_URL}, links SOME_URL to the url SOME_URL
\providecommand*\url[1]{\href{#1}{#1}}
% Same as above, but pretty-prints SOME_URL in teletype fixed-width font
\renewcommand*\url[1]{\href{#1}{\texttt{#1}}}

% For \email{ADDRESS}, links ADDRESS to the url mailto:ADDRESS
\providecommand*\email[1]{\href{mailto:#1}{#1}}
% Same as above, but pretty-prints ADDRESS in teletype fixed-width font
%\renewcommand*\email[1]{\href{mailto:#1}{\texttt{#1}}}

%\providecommand\BibTeX{{\rm B\kern-.05em{\sc i\kern-.025em b}\kern-.08em
%    T\kern-.1667em\lower.7ex\hbox{E}\kern-.125emX}}
%\providecommand\BibTeX{{\rm B\kern-.05em{\sc i\kern-.025em b}\kern-.08em
%    \TeX}}
\providecommand\BibTeX{{B\kern-.05em{\sc i\kern-.025em b}\kern-.08em
    \TeX}}
\providecommand\Matlab{\textsc{Matlab}}

% Custom hyphenation rules for words that LaTeX has trouble with
\hyphenation{bio-mim-ic-ry bio-in-spi-ra-tion re-us-a-ble pro-vid-er}

%%%%%%%%%%%%%%%%%%%%%%%% End Helper Commands %%%%%%%%%%%%%%%%%%%%%%%%%%%

%%%%%%%%%%%%%%%%%%%%%%%%% Begin CV Document %%%%%%%%%%%%%%%%%%%%%%%%%%%%
\fancypagestyle{firststyle}
{
 %  \fancyhf{ss}
%\renewcommand{\footrulewidth}{0.4pt}% default is 0pt
%\renewcommand{\footruleheight}{0.4pt}% default is 0pt
   \fancyfoot[L]{ \line(1,0){144} \\
%\textsuperscript{$\dagger$} \textbf{SNePS} is a knowledge representation and reasoning system, developed at Buffalo University, NY.\\
%\textsuperscript{$\dagger$} In Egypt, the \textbf{M.Sc.} of engineering is a \textbf{1 - 2 year} research study, that must be \textit{preceded} by a \textbf{B.Sc.} of\\ ~~engineering, which consists of \textbf{9} semesters  of courses and \textbf{6} months of research.
\textsuperscript{$\dagger$} In Egypt, the M.Sc. of engineering is a 1 - 2 year research study, preceded by a 5 year engineering B.Sc.
}
}

\fancypagestyle{secondstyle}
{
%\fancyhead[L]{\hspace{46 mm}2 of 3}
 %  \fancyhf{ss}
%\renewcommand{\footrulewidth}{0.4pt}% default is 0pt
%\renewcommand{\footruleheight}{0.4pt}% default is 0pt
   \fancyfoot[L]{ \line(1,0){144} \\
\textsuperscript{$\S$} See the footnote regarding the \textbf{M.Sc.} degree on the first page.\\
%\vspace{1.5pt}
%\hspace{77 mm}2 of 3
}
%\textit{\line(1,0){144} \\ Please note that this is a CV directed to the academia. For a \textbf{professional} CV, please e-mail the author.}
%477
}


\fancypagestyle{thirdstyle}
{
\fancyhead[L]{\hspace{46 mm}3 of 3}
  \fancyfoot[L]{}
}
%\usepackage{showframe}


\begin{document}
\makeheading{Nasr Salim Kasrin (MSc., Eng., BSc.)}

\section{Personal}

% NOTE: Mind where the & separators and \\ breaks are in the following
%       table. Table is one row made up of three parboxes. The left
%       parbox has address info, the middle parbox has a vertical bar,
%       and the right parbox has phone and electronic contact
%       information.
%
% MACROS: \rcollength is the width of the right column of the table
%             (adjust it to your liking; default is 1.85in).
%         \spacewidth is width of area between left and right boxes.
%         \spacechar is character used to produce perforated vertical
%             boundary between boxes.
%
\newlength{\rcollength}\setlength{\rcollength}{2.5in}%
\newlength{\spacewidth}\setlength{\spacewidth}{55pt}
\newcommand\spacechar{$|$}
%
\begin{tabular}[t]{@{}p{\textwidth-\rcollength-\spacewidth}@{}p{\spacewidth}@{}p{\rcollength}}%

% Address box
\parbox{\textwidth-\rcollength-\spacewidth}{%
Address ~~~~~~Bamberg, Germany\\
Birthdate ~~~~Jan. 1986\\
Nationality ~~Syrian
%Telephone ~~~+49--17636642113
}

% Cheesy perforated vertical bar between boxes
% Shorten by removing \spacechar's
& \parbox{\spacewidth}{\centering \spacechar\\\spacechar\\\spacechar\\\spacechar} &

% Non-snail-mail contact information
\parbox{\rcollength}{%
E-mail~~~~~\email{nasr.kasrin@gmail.com}\\
LinkedIn ~\href{https://www.linkedin.com/in/nkasrin/}{linkedin.com/in/nkasrin}\\
DBLP ~~~~~\href{https://dblp.org/pid/30/8393.html}{dblp.org/pid/30/8393.html}
}

\end{tabular}

%\hline

%%
%% In modern CV's, it seems like ``Objective'' is frowned upon. Instead,
%% incorporate it into a well-constructed cover letter. The ``More
%% information'' can go at the end of the CV, but it should not distract
%% from the section giving references available to contact.
%%
%
% \section{Objective}
%
% Placement in an academic position (i.e., faculty, postdoctoral, or
% research scientist) that allows for advanced research in distributed
% complex adaptive systems (i.e., modeling, analysis, design, and
% verification) with a particular focus on the control of engineered
% agents (e.g., for communications, control, software, electronics, and
% sustainability) and the analysis of biological phenomena (e.g.,
% self-organization, ecological rationality)
% \begin{innerlist}
% \item More information and auxiliary documents can be found at\\\url{http://www.tedpavlic.com/facjobsearch/}
% \end{innerlist}


%\section{Profile}

%I am a computer science and digital media professional with academic and industry experience in Artificial Intelligence, Human-Computer Interaction, and software engineering. I researched and taught AI and developed AI solutions for commercial products. For HCI and software engineering, I have been involved in the design and development of several commercial mobile applications in addition to teaching courses on the topics. \vspace{1.5 mm}

%I am a computer science and digital media professional with academic and industry experience in software engineering and computer systems, Artificial Intelligence and Human-Computer Interaction. I served as a technical team leader in the design and development of several commercial applications, as an R\&D solution developer, initiated and led the development educational software tools as well as AI software agents. I have also taught programming courses and courses on the analysis and design of software.\vspace{1.5 mm}

%I am currently assistant dean and lecturer in the Multimedia \& Creative Technologies faculty and lecturer in the Business IT (BIT) faculty at the International University of Technology, Twin Tech in Yemen where I teach both technical courses and some applied arts courses in addition to hybrid courses like \textit{HCI} for the Multimedia faculty and \textit{Information Design and Visualization}. \vspace{1.5 mm}

%I am a dynamic person who takes a lot of initiative and is always committed to mastering what I am working on. I enjoy adapting to new challenges and to be influenced by and influence my environment positively.

\section{Education}
%\textsuperscript{$\dagger$}

\textit{PhD}, Computer Science Subject Group, \hfill Jan. 2016 -- 2022 (\textit{expected})\\
\href{https://www.uni-bamberg.de/wiai/}
{Faculty of Information Systems and Applied Computer Sciences},\\ \href{https://www.uni-bamberg.de/}{University of Bamberg}, Bamberg, Germany \vspace{3 mm}

\textit{MSc}, Computer Science \& Engineering Department, \hfill Jan. 2009 -- Sep. 2010\\
        \href{http://met.guc.edu.eg/}
			{Faculty of Media Engineering \& Technology},\\ \href{http://guc.edu.eg/}{German University in Cairo (GUC)}, Egypt \vspace{3 mm}

\textit{BSc}, Computer Science \& Engineering Department, \hfill Sep. 2003 -- Jul. 2008\\
        \href{http://met.guc.edu.eg/}
			{Faculty of Media Engineering \& Technology},\\ \href{http://guc.edu.eg/}{German University in Cairo (GUC)}, Egypt\vspace{2.5 mm}

%		 ~~~~~Thesis Grade: \textit{Excellent}; Overall Grade: \textit{Very Good}

\section{Academic\\Experience}

\textbf{Research Associate (Third-party funded projects)} \hfill{Oct. 2015 -- Jan. 2021}
\begin{outerlist}
	\item[] Chair of Computer Science, Mobile Software Systems / Mobility at Otto-Friedrich-University, Bamberg, Germany.
	\begin{innerlist}
		\item Follow up on the SIMUTOOL EU Project, with team size from 2 to 5 members including research assistants, students assistants, and masters theses.
		\item Research and development tasks including analyzing requirements and developing solutions for the domain.
		\item Product owner of a knowledge management platform developed for the manufacturing industry.
		\item Managed a total of 5 software developers at different points to develop interrelated components.
		\item Represented the group at research meetings with industry partners and meeting with EC officials.\\
		%       \item Teaching  and \textit{Design Thinking \& Innovation} in the Business Information Technology faculty. \textit{See lecturer responsibilities below}.\\
	\end{innerlist}
\end{outerlist}



\textbf{Acting Dean} \hfill{Oct. 2014 -- Sep. 2015}
\begin{outerlist}
    \item[] \href{http://iutt.edu.ye/}{Faculty of Multimedia and Creative Technologies \& Faculty of Business IT (BIT) The International University of Technology Twintech}, Sana'a, Yemen.
    \begin{innerlist}
       \item Course (curriculum) leader for the Multimedia faculty. Responsibilities include monitoring the quality of the implementation of the faculty's courses.
       \item Board member of the curriculum development team for the faculties of Multimedia and Business IT.
       \item Courses taught: Object Oriented Programming, Information Design and Visualization, Object Oriented Analysis and Design of Software, Art \& Design History, Art \& Design History, Design Thinking \& Innovation.\\
    \end{innerlist}
\end{outerlist}


\textbf{Lecturer} \hfill{Nov. 2013 -- Sep. 2014}
\begin{outerlist}
    \item[] \href{http://iutt.edu.ye/}{Faculty of Business Information Technology, The International University of Technology Twintech}, Sana'a, Yemen.
    \begin{innerlist}
        \item Delivered lectures and tutorials; developed teaching material; prepared and evaluated exams, assignments, quizzes, and projects. Supervised graduation projects.
        \item Courses taught: Human-Computer Interaction, Object-Oriented Analysis \& Design, Object-Oriented Programming, and Research \& Development.\\
    \end{innerlist}
\end{outerlist}

%\newpage

\textbf{Teaching Assistant} \hfill{Oct. 2008 -- Aug. 2012}
\begin{outerlist}
    \item[] \href{http://met.guc.edu.eg/}{Department of Computer Science and Engineering},
            \href{http://guc.edu.eg/}{GUC}, Cairo, Egypt.
    \begin{innerlist}
        \item Prepared and taught tutorial sessions for: \href{http://met.guc.edu.eg/Courses/CourseEdition.aspx?crsEdId=278}{Introduction to Artificial Intelligence}, \href{http://met.guc.edu.eg/Courses/CourseEdition.aspx?crsEdId=271}{Introduction to Computer Science}, \href{http://met.guc.edu.eg/Courses/CourseEdition.aspx?crsEdId=308}{Introduction to Computer Programming}.
       \item Prepared and taught 10+ hours per week of tutorial sessions.
       \item Evaluated and collaborated on preparing assignments, quizzes, and projects.
       \item Served as academic advisor for more than 100 students to help guide them in planning  and carrying out their study program up to graduation.\\
    \end{innerlist}
\end{outerlist}


\textbf{Junior Teaching Assistant} \hfill {Feb. 2005 -- Jun. 2005}
\begin{outerlist}
    \item[] \href{http://met.guc.edu.eg/}{Department of Computer Science and Engineering},
            \href{http://guc.edu.eg/}{GUC}, Cairo, Egypt.
    \begin{innerlist}
        \item Assisted teaching assistants in carrying out labs of 25 students.
    \end{innerlist}
\end{outerlist}
%\thispagestyle{secondstyle}


\section{Industry\\Experience}

\textbf{Team Leader, Software Architect} \hfill {Sep. 2012 -- Dec. 2013}
\begin{outerlist}
    \item[] \href{https://web.archive.org/web/20160704123950/http://www.tayait.com/#}{Taya IT}, Cairo, Egypt
    \begin{innerlist}
	\item Planned and supervised the iterative  development of \href{https://itunes.apple.com/us/app/greetings-studio-personalized/id495706159?mt=8}{Greetings Studio (GS)}, a social networking app for creating and sharing e-cards and visual sentiments.
	\item Redesigned the architecture of GS for scalability and de-coupling of components.
	\item Contributed in setting and developing the business strategy of GS.
	\item Collaborated with the user-experience team on product design and development.\\
    \end{innerlist}
\end{outerlist}

%\newpage

\textbf{Research \& Development Engineer} \hfill {Apr. 2011 -- Aug. 2012}
\begin{outerlist}
    \item[] \href{https://web.archive.org/web/20160704123950/http://www.tayait.com/#}{Taya IT}, Cairo, Egypt
    \begin{innerlist}
      \item Led a team to develop computer players for the iOS backgammon game \href{https://itunes.apple.com/us/app/tawla-backgammon-game-arabian/id441574632?mt=8}{Tawla}.
      \item Designed and developed algorithms for several software products.
      \item Investigated and collaborated on UX/UI and product design and development.
      \item Investigated game design and collaborated in the production of the 3D physics-based puzzle game \href{https://itunes.apple.com/us/app/save-bomb-addictive-challenging/id532135849?mt=8}{Save The Bomb}.\\
%      \item Investigated trends in \textit{semantic web} and \textit{information architecture}.\\
	\end{innerlist}
\end{outerlist}


\textbf{Information Retrieval \& Extraction Developer} \hfill {Jul. 2005 -- Nov. 2005}
\begin{outerlist}
    \item[] Alzoa.com. An Arabic news portal (Currently offline; \href{http://web.archive.org/web/20120310211501/http://www.alzoa.com/}{archived version})
    \begin{innerlist}
		\item Assisted in the development of an Arabic search engine using Lucene in Java.
		\item Built web-crawlers (using Perl) to collect and archive articles from news websites.
    \end{innerlist}
\end{outerlist} 

\section{Language}
\textbf{Programming Languages}\vspace{1mm}

High Proficiency: Python.\vspace{1mm}


Past Proficiency: Java, lisp (functional), perl, prolog, microcontroller programming, C++.\vspace{2mm}


\textbf{Spoken Languages}\vspace{1mm}

Arabic:~~~Native

English: ~Fluent (Full proficiency in reading, writing, and speaking)

German: Basic ~(A1)


%\textbf{Software Development}\vspace{1mm}
%
%~~~Object-Oriented Analysis and Design, UML, Iterative Software Development.\vspace{2mm}
%
%\textbf{Technologies / OSs}\vspace{1mm}
%
%~~~iOS Technologies, Windows, Mac.


\section{Publications}

\begin{bibenum}
	
	\item \textbf{Nasr Kasrin}, et al.: Data-Sharing Markets for Integrating IoT Data Processing Functionalities,  CCF Trans. Pervasive Comp. Interact. 3, 76–93 (2021).
	
    \item \textbf{Nasr Kasrin}, et al.: Semantic Data Management for Experimental Manufacturing Technologies, \emph{Datenbank-Spektrum 18(1): 27-37}, 2018. 
	
    \item Haythem O. Ismail and \textbf{Nasr Kasrin}: \href{http://met.guc.edu.eg/Repository/Faculty/Publications/317/KI2010-62.pdf}{Focused Belief Revision as a Model of Fallible Relevance-Sensitive Perception} , \emph{33rd German AI Conference KI}, 2010. 

    \item Haythem O. Ismail and \textbf{Nasr Kasrin} : \href{http://met.guc.edu.eg/Repository/Faculty/Publications/319/ECAI-510.pdf}{High-Level Perception as Grounded Focused Belief Revision}, \emph{European Conference on Artificial Intelligence ECAI}, 2010. 

    \item Ahmed Y. Tawfik and \textbf{Nasr Kasrin} : Integrating Causal Knowledge in Case-based Retrieval: Causal Decomposition of Cases , \emph{In Proceedings of the Thirteenth UK Workshop on Case Based Reasoning UKCBR}, 2008.
\end{bibenum}

%\section{Conferences}
%Attended and presented research findings at ECAI 2010, Lisbon, Portugal. 

%\section{Presentations}
%
%``\textbf{Planning: A Hard Problem?}  \textit{Past to Present}'' (2011).  A look at popular planning systems in AI, comparing their strengths and limitations. \textit{Presented to the Computational Logic and Cognitive Systems research group (CLoCS) at the GUC}, Cairo, Egypt.\\
%
%``\textbf{The Muse in the Machine:} \textit{Artificial Creativity and Art}'' (2008). A lecture on algorithmic / computational art with a case study on algorithmic composition. \textit{Presented in the multi-disciplinary lecture series ``Mathematics and Art'' at the GUC}, Cairo, Egypt.

%``\textbf{Java3D}'' (2005). A lecture on Java3D technology, which was still under-documented and with a small online community at the time. \textit{Presented at the GUC}, Cairo, Egypt. %

% Add a little space to nudge next ``Conference Publications'' marginpar
% down to make room for tall ``Submitted Journal Publications''
% marginpar. If there are enough submitted journal publications, this
% space will not be needed (and should be removed).
%\vspace{0.1in}


%\thispagestyle{thirdstyle}


% $~~\bullet$     Regularly participated in meetings about computational logic.


%\textit{Very good analytical skills}. Have the ability to think (very) abstractly when needed and not to think pragmatically when needed.\\

%\textit{Very good communication of knowledge and presentation skills}. Gave a couple of public lectures about topics such as: Artificial Intelligence and ``Artificial Creativity and Art'', the second of which contained some original research. Have given 14 hours of tutorials every week for the last four years. I have gained experience on the art and craft of presentation and communication of information and knowledge. I have gave presentations and tutorials to audience of ranging backgrounds: first year university students in introductory courses, 5th year university students in pre-masters courses, and some public lectures which contained professors from fields such as Art, Mathematics and computer science.\\

%\textit{Very good writing skills}. I have very good technical writing skills, and very good literary writing skills. I have written a technical Master's thesis which included disciplines such as Computer Science, Cognitive Science, Analytical Philosophy, and Mathematics. Additionally, have been a co-author of three technical papers published in worldwide peer-reviewed conferences and many technical blog articles and short essays. For literary skills, I have been consistently active in writing poetry for the last 8 years, and have occasionally written short stories and film scripts.\\


% The ``More Info'' section may not be necessary; make sure it's short
% so it doesn't prevent people from seeing references available to
% contact.


\section{References}
Professional as well as academic references are furnished upon request.
\end{document}

%%%%%%%%%%%%%%%%%%%%%%%%%% End CV Document %%%%%%%%%%%%%%%%%%%%%%%%%%%%%

%----------------------------------------------------------------------%
% The following is copyright and licensing information for
% redistribution of this LaTeX source code; it also includes a liability
% statement. If this source code is not being redistributed to others,
% it may be omitted. It has no effect on the function of the above code.
%----------------------------------------------------------------------%
% Copyright (c) 2007, 2008, 2009, 2010, 2011 by Theodore P. Pavlic
%
% Unless otherwise expressly stated, this work is licensed under the
% Creative Commons Attribution-Noncommercial 3.0 United States License. To
% view a copy of this license, visit
% http://creativecommons.org/licenses/by-nc/3.0/us/ or send a letter to
% Creative Commons, 171 Second Street, Suite 300, San Francisco,
% California, 94105, USA.
%
% THE SOFTWARE IS PROVIDED "AS IS", WITHOUT WARRANTY OF ANY KIND, EXPRESS
% OR IMPLIED, INCLUDING BUT NOT LIMITED TO THE WARRANTIES OF
% MERCHANTABILITY, FITNESS FOR A PARTICULAR PURPOSE AND NONINFRINGEMENT.
% IN NO EVENT SHALL THE AUTHORS OR COPYRIGHT HOLDERS BE LIABLE FOR ANY
% CLAIM, DAMAGES OR OTHER LIABILITY, WHETHER IN AN ACTION OF CONTRACT,
% TORT OR OTHERWISE, ARISING FROM, OUT OF OR IN CONNECTION WITH THE
% SOFTWARE OR THE USE OR OTHER DEALINGS IN THE SOFTWARE.
%----------------------------------------------------------------------%