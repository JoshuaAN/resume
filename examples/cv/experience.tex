%-------------------------------------------------------------------------------
%	SECTION TITLE
%-------------------------------------------------------------------------------
\cvsection{Experience}


%-------------------------------------------------------------------------------
%	CONTENT
%-------------------------------------------------------------------------------
\begin{cventries}

%---------------------------------------------------------
  \cventry
    {Radboud University Medical Center Nijmegen RUNMC} % Organization
    {Identification and ranking of p63 binding sites putatively
    involved in the etiology of non-syndromic cleft lip with or
    without cleft palate.} % Job title
    {Nijmegen, Netherlands} % Location
    {2017} % Date(s)
    {
      \begin{cvitems} % Description(s) of tasks/responsibilities
        \item {Took ownership of an initially wet-lab project and transformed it in a }
        \item {In silico prediction and in-vitro validation of clinically relevant transcription factor binding sites using a integrative multiomics approach.}
        \item {Development of a pipeline to integrate publicly available and in-house multi-omics data (Chip-seq, SNPs, GWAS, linkage disequilibrium).}
        \item {R/bioconductor because of a lack of bioinformatic expertise in the research group.}
      \end{cvitems}
    }

%---------------------------------------------------------
  \cventry
    {Center for Molecular and Biomolecular Informatics CMBI} % Organization
    {CTCF-motif directionality controls CTCF-mediated chromatin interactions
    and correlates with topological domain structure.} % Job title
    {Nijmegen, Netherlands} % Location
    {Jan. 2016 - Jun. 2017} % Date(s)
    {
      \begin{cvitems} % Description(s) of tasks/responsibilities
        \item {Drove and transformed a loosely defined, explorative research project in a hypothesis driven project resulting in a publication.}
        \item {Hypothesis generation by leveraging multiomics datasets describing different dimensionalities of the genome, ranging 1D (sequence), 2D (ChIP-Seq), and 3D data (ChIA-PET, HI-C).}
        \item {Hypothesis testing by applying parametric and non-parametric methods, randomization, as well as modeling of chromatin loops.}
        \item {Apllying unsupervised machine learning techniques e.g. PCA, HCA as well as a multitude of visualizations for data exploration.}
      \end{cvitems}
    }
%---------------------------------------------------------
\end{cventries}
