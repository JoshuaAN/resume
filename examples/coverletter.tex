%!TEX TS-program = xelatex
%!TEX encoding = UTF-8 Unicode
% Awesome CV LaTeX Template for Cover Letter
%
% This template has been downloaded from:
% https://github.com/posquit0/Awesome-CV
%
% Authors:
% Claud D. Park <posquit0.bj@gmail.com>
% Lars Richter <mail@ayeks.de>
%
% Template license:
% CC BY-SA 4.0 (https://creativecommons.org/licenses/by-sa/4.0/)
%


%-------------------------------------------------------------------------------
% CONFIGURATIONS
%-------------------------------------------------------------------------------
% A4 paper size by default, use 'letterpaper' for US letter
\documentclass[11pt, a4paper]{awesome-cv}

% Configure page margins with geometry
\geometry{left=1.4cm, top=.8cm, right=1.4cm, bottom=1.8cm, footskip=.5cm}

% Specify the location of the included fonts
\fontdir[fonts/]

% Color for highlights
% Awesome Colors: awesome-emerald, awesome-skyblue, awesome-red, awesome-pink, awesome-orange
%                 awesome-nephritis, awesome-concrete, awesome-darknight
\colorlet{awesome}{awesome-red}
% Uncomment if you would like to specify your own color
% \definecolor{awesome}{HTML}{CA63A8}

% Colors for text
% Uncomment if you would like to specify your own color
% \definecolor{darktext}{HTML}{414141}
% \definecolor{text}{HTML}{333333}
% \definecolor{graytext}{HTML}{5D5D5D}
% \definecolor{lighttext}{HTML}{999999}

% Set false if you don't want to highlight section with awesome color
\setbool{acvSectionColorHighlight}{true}

% If you would like to change the social information separator from a pipe (|) to something else
\renewcommand{\acvHeaderSocialSep}{\quad\textbar\quad}


%-------------------------------------------------------------------------------
%	PERSONAL INFORMATION
%	Comment any of the lines below if they are not required
%-------------------------------------------------------------------------------
% Available options: circle|rectangle,edge/noedge,left/right
% \photo[circle,noedge,left]{./examples/profile}
\name{Jonas}{Falck}
\position{Bioinformatics{\enskip\cdotp\enskip}Data Analysis}
\address{Muzenplaats 4, 6525JA, Nijmegen, Netherlands}

\mobile{(+31) 655707314}
\email{jonas@falcken.de}
%\homepage{www.posquit0.com}
\github{joe-nas}
%\linkedin{posquit0}
% \gitlab{gitlab-id}
% \stackoverflow{SO-id}{SO-name}
% \twitter{@twit}
% \skype{skype-id}
% \reddit{reddit-id}
% \medium{madium-id}
% \googlescholar{googlescholar-id}{name-to-display}
%% \firstname and \lastname will be used
% \googlescholar{googlescholar-id}{}
% \extrainfo{extra informations}

% \quote{``rtfm"}


%-------------------------------------------------------------------------------
%	LETTER INFORMATION
%	All of the below lines must be filled out
%-------------------------------------------------------------------------------
% The company being applied to
\recipient
  {Assistant Professor dr.ir. Bernd Brandt}
  {Academic Center for Dentistry Amsterdam\\The Netherlands}
% The date on the letter, default is the date of compilation
\letterdate{\today}
% The title of the letter
\lettertitle{Job Application for data engineer for high-dimensional data in relation to oral health}
% How the letter is opened
\letteropening{Dear dr. ir. Bernd Brandt}
% How the letter is closed
\letterclosing{Sincerely,}
% Any enclosures with the letter
%\letterenclosure[Attached]{Curriculum Vitae}


%-------------------------------------------------------------------------------
\begin{document}

% Print the header with above personal informations
% Give optional argument to change alignment(C: center, L: left, R: right)
\makecvheader[R]

% Print the footer with 3 arguments(<left>, <center>, <right>)
% Leave any of these blank if they are not needed
\makecvfooter
  {\today}
  {Jonas Falck~~~·~~~Motivation}
  {}

% Print the title with above letter informations
\makelettertitle

%-------------------------------------------------------------------------------
%	LETTER CONTENT
%-------------------------------------------------------------------------------
\begin{cvletter}

\lettersection{About Me}
After my civil service duty, I moved from Germany to the Netherlands in order to study medical biology in Nijmegen.
After finishing my bachelor studies, I started my master's in medical biology - medical epigenomics track.
For financial and personal reasons, I could not finish my master studies with a diploma.
However, I finished the mandatory two internships - during which I taught myself bioinformatics - and the bulk of the courses.
I started working in logistics to finance myself and my studies.
What was supposed to be a temporary solution to my financial situation turned out to be detrimental to the remainder of my study and to be a longer endeavour - lasting 4 years.
Although having a lot of nice colleagues and superiors, who value me as a person and employee, I want a change in my life and start following my passion again.
I want to work in a field that I am deeply interested in, and which is more meaningful to me.
\lettersection{Why The Department of Preventive Dentistry at ACTA and Data Science Centre of the University of Amsterdam DSC UvA?}
I believe at ACTA and DSC UvA, I could pursue my passion and get back working in the field of next-generation sequencing and data-analysis and to finally start off my career in bioinformatics.
Within the Personal Microbiome Health (PMH) research priority area, as well as for the METAHEALTH project, I could put my skills to use and implement user friendly standard operating procedures (SOPs) for next-generation sequencing data-handling and analysis, while delving into scalable frameworks for reproducible data analysis like nextflow, snakemate or cwl.
Given that I am also interested in data-science and modelling, I like the idea of gathering and integrating data from different sources in order to model oral health.
At ACTA within PMH and the DSC UvA I could share and discuss my work within a highly motivated and knowledgeable interdisciplinary group of peers and I would be in the best environment to learn and expand my skills.
\lettersection{Why Me?}
I believe that I am the ideal candidate for this job, because I have proven technical skills to perform and analyse omics data. 
I worked with publicly available data and I strongly agree with FAIR data principles. 
Furthermore, my internships and especially working in logistics taught me that well implemented SOPs greatly improve performance and productivity. 
As such I believe that SOPs are fundamental to reproducible research and meaningful results. 
Omics and the ability to study a large population of molecules at once, the vast amount of data generated and the technical as well as analytical intricacies this comes with fascinate me and I am keen on learning the ins and outs of metagenomic analysis. 
I am proud of my autodidacticism which allowed me to develop most of the skills presented in my CV. 
After all, I don't like idling, I am a hands-on and down to earth person as well as a hard worker. 
I'd be pleased if you consider my application.
\end{cvletter}


%-------------------------------------------------------------------------------
% Print the signature and enclosures with above letter informations
\makeletterclosing

\end{document}
