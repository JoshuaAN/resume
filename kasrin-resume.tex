%!TEX TS-program = xelatex
%!TEX encoding = UTF-8 Unicode
% Awesome CV LaTeX Template for CV/Resume
%
% This template has been downloaded from:
% https://github.com/posquit0/Awesome-CV
%
% Author:
% Claud D. Park <posquit0.bj@gmail.com>
% http://www.posquit0.com
%
% Template license:
% CC BY-SA 4.0 (https://creativecommons.org/licenses/by-sa/4.0/)
%


%-------------------------------------------------------------------------------
% CONFIGURATIONS
%-------------------------------------------------------------------------------
% A4 paper size by default, use 'letterpaper' for US letter
\documentclass[12pt, a4paper]{awesome-cv}

% Configure page margins with geometry
\geometry{left=1.4cm, top=.8cm, right=1.4cm, bottom=1.8cm, footskip=.5cm}

% Color for highlights
% Awesome Colors: awesome-emerald, awesome-skyblue, awesome-red, awesome-pink, awesome-orange
%                 awesome-nephritis, awesome-concrete, awesome-darknight
\colorlet{awesome}{awesome-orange}
% Uncomment if you would like to specify your own color
%\definecolor{awesome}{HTML}{f23a49}

% Colors for text
% Uncomment if you would like to specify your own color
% \definecolor{darktext}{HTML}{414141}
% \definecolor{text}{HTML}{333333}
% \definecolor{graytext}{HTML}{5D5D5D}
% \definecolor{lighttext}{HTML}{999999}
% \definecolor{sectiondivider}{HTML}{5D5D5D}

% Set false if you don't want to highlight section with awesome color
\setbool{acvSectionColorHighlight}{true}

% If you would like to change the social information separator from a pipe (|) to something else
\renewcommand{\acvHeaderSocialSep}{\quad\textbar\quad}




%-------------------------------------------------------------------------------
% USER DEFINED (NK)


\definecolor{offwhite}{RGB}{251,247,245}

\pagecolor{offwhite}

\definecolor{lblue}{RGB}{59, 83, 239} 
%\definecolor{lblue}{RGB}{58, 73, 242} 


\newcommand{\see}[1]{\textit{(See \urll{#1} for more)}}
\newcommand{\urll}[1]{\textcolor{lblue}{\underline{\href{https://#1}{#1}}}}


% END OF USER DEFINED
%-------------------------------------------------------------------------------






%-------------------------------------------------------------------------------
%	PERSONAL INFORMATION
%	Comment any of the lines below if they are not required
%-------------------------------------------------------------------------------
% Available options: circle|rectangle,edge/noedge,left/right
\photo[rectangle,edge,right]{./pic-600-box.jpg}
\name{Nasr}{Kasrin (PhD)}
\position{Software Architect{\enskip\cdotp\enskip}Product Thinker}


\address{96050 Bamberg, Germany}


%\born{January 1986}

\mobile{(+49) 176-3664-2113}                  
\email{n42r.me@gmail.com}
%\social[linkedin]{nkasrin}                         % optional, remove / comment the line if not wanted
%\social[github]{n42r}                              % optional, remove / comment the line if not wanted

%\phone[fixed]{+2~(345)~678~901}
%\phone[fax]{+3~(456)~789~012}

\homepage{n42r.github.io}



%\mobile{(+82) 10-9030-1843}
%\email{posquit0.bj@gmail.com}
%\dateofbirth{January 1st, 1970}
%\homepage{www.posquit0.com}
%\github{posquit0}
%\linkedin{posquit0}
% \gitlab{gitlab-id}
% \stackoverflow{SO-id}{SO-name}
% \twitter{@twit}
% \skype{skype-id}
% \reddit{reddit-id}
% \medium{madium-id}
% \kaggle{kaggle-id}
% \hackerrank{hackerrank-id}
% \googlescholar{googlescholar-id}{name-to-display}
%% \firstname and \lastname will be used
% \googlescholar{googlescholar-id}{}
% \extrainfo{extra information}

%\quote{``Be the change that you want to see in the world."}


%-------------------------------------------------------------------------------
\begin{document}

% Print the header with above personal information
% Give optional argument to change alignment(C: center, L: left, R: right)
\makecvheader[C]

% Print the footer with 3 arguments(<left>, <center>, <right>)
% Leave any of these blank if they are not needed
%\makecvfooter
%  {\today}
%  {Byungjin Park~~~·~~~Résumé}
%  {\thepage}


%-------------------------------------------------------------------------------
%	CV/RESUME CONTENT
%	Each section is imported separately, open each file in turn to modify content
%-------------------------------------------------------------------------------
%-------------------------------------------------------------------------------
%	SECTION TITLE
%-------------------------------------------------------------------------------

\cvsection{Summary}
%\vspace{2ex}


%-------------------------------------------------------------------------------
%	CONTENT
%-------------------------------------------------------------------------------
\begin{cvparagraph}

%---------------------------------------------------------

Dynamic software architect and product strategist with a history of guiding successful projects spanning various domains (including data architecture, social networks, mobile apps, games, robotics, and AI), team sizes ranging from 2 to 10 members, and diverse settings, from agile B2C environments to R\&D projects backed by €3.5 million in public funding.

With expertise at the intersection of engineering and product leadership, I excel in delivering cutting-edge solutions that drive business growth and enhance user experiences \see{n42r.github.io}.

\end{cvparagraph}

\cvsection{Projects}

\begin{cventries}
  \cventry
    {} {Hackathons} {} {} {
      \begin{cvitems}
      \item {Built device costing <\$150 to control and monitor set top boxes replacing some need for devices costing >\$1000.}
        \begin{itemize}
        \item {Python script accessing infrared transeiver on Raspberry Pi's GPIO to control set-top box.}
        \item {Monitored data to verify channel activity}
        \end{itemize}
      \end{cvitems}
    }

  \cventry
    {} {Hydra} {} {} {
      \begin{cvitems} % Description(s) bullet points
      \item {Moved services from single physical sever to individual docker containers: web ui, web api, web proxy, beanstalkd, database}
      \item {Wrote ansible playbook to spin up host instance and all service containers in OpenStack from job in Jenkins}
      \end{cvitems}
    }

  \cventry
    {} {Wifi} {} {} {
      \begin{cvitems} % Description(s) bullet points
      \item {Built out test bed containing wifi controller, APs, and networking gear}
      \item {Wrote Robot Framework upgrade tests}
      \item {Tests drove wifi controller via web ui using customer workflow}
      \item {Integrated tests into Jenkins automation pipeline}
      \end{cvitems}
    }

  \cventry
    {} {Vulture} {} {} {
      \begin{cvitems} % Description(s) bullet points
      \item {Wrote Python log collection utitily for postmortem test failure analysis}
      \item {Collected device logs, cli command output, and automated Robot Framework test logs}
      \end{cvitems}
    }

\end{cventries}

%-------------------------------------------------------------------------------
%	SECTION TITLE
%-------------------------------------------------------------------------------
\clearpage 
\cvsection{Work Experience}


%-------------------------------------------------------------------------------
%	CONTENT
%-------------------------------------------------------------------------------
\begin{cventries}

%---------------------------------------------------------


  \cventry
    {University of Bamberg (Third-party Funded Project)} % Job title
    {Research Associate (Architect | Team Lead)} % Organization
    {2015 - 2020} % Location
    {Bamberg, Germany} % Date(s)
    {
      \begin{cvitems} % Description(s) of tasks/responsibilities
		\item {Directed a 4-year project to develop a data management SaaS for a €3.5 million 8-company EU manufacturing project, optimizing data discovery, collaboration, and turnaround time, and resulting in enhanced operational efficiency \see{github.com/simutool}.}
		\item {Cultivated close relationships with 10+ external partners, facilitating deep domain understanding and precise requirements identification.}
		\item {Engineered a read-heavy, horizontally scalable SaaS, ensuring seamless operations and future-proof architecture (ex., stateless nodes).}
      \end{cvitems}
    }

%---------------------------------------------------------


%  \cventry
%    {University of Technology Twintech} % Job title
%    {Assistant Dean / Lecturer} % Organization
%    {2013 - 2015} % Location
%    {Sanaa, Yemen} % Date(s)
%    {
%      \begin{cvitems} % Description(s) of tasks/responsibilities
%		\item {Managed the IT and Multimedia faculties, reporting to the university president}
%		\item {Modernized the curriculum by leading a curriculum development initiative for the Business IT and Multimedia faculties}
%		\item {Taught several Courses: \emph{Software Analysis}, \emph{Human-Computer Interaction}, \emph{Information Design}, etc}
%      \end{cvitems}
%    }

%---------------------------------------------------------


  \cventry
    {TayaIT} % Job title
    {Team Leader | Software Architect | R\&D Engineer} % Organization
    {2011 - 2014} % Location
    {Cairo, Egypt} % Date(s)
    {
      \begin{cvitems} % Description(s) of tasks/responsibilities
		\item {Directed agile technical teams, ranging from 2 to 5 members, in the development of two enduring social media/mobile products, driving perpetual augmentation of app rankings (4.5 starts) and a 10-fold increase in user engagement (See 'Greetings Studio' and 'Tawla' in Projects).}
		\item {Reduced feedback-development cycle times by 25\% by coordinating cross-functional collaboration between technical, business, and UI/UX teams, streamlining workflows and fostering tighter cooperation and heightened productivity.}
		\item {Saved the company over 10-man months by investigating emerging technologies and alternative project paths and advising the CEO in adopting better paths or avoiding dead-ends and sub-optimum paths.}
      \end{cvitems}
    }

%---------------------------------------------------------


%\cventry{2008 -- 2012}{Teaching Assistant}{The German University in Cairo (GUC)}{Cairo}{Egypt}{
%\begin{itemize}
%\item Taught 10 hours per week of tutorials, interfacing with 150 new students per semester.
%\item Courses: \emph{Introduction to Artificial Intelligence}, \emph{Introduction to Computer Science}, \emph{Introduction to Computer Programming}. 
%\end{itemize}}


\end{cventries}

%-------------------------------------------------------------------------------
%	SECTION TITLE
%-------------------------------------------------------------------------------
\cvsection{Skills}


%-------------------------------------------------------------------------------
%	SUBSECTION TITLE
%-------------------------------------------------------------------------------
%\cvsubsection{Databases}


%-------------------------------------------------------------------------------
%	CONTENT
%-------------------------------------------------------------------------------
\begin{cvskills}


\cvskill{DEVELOPMENT}{Python, Docker (multi-container), Linux CLI, git, and low-/no-code (bubble.io). Formerly, C++, Lisp, Java.}

\cvskill{DATABASES}{MongoDB, Neo4j, SQL, Google Firebase, and Resource Description Framework (RDF) / Semantic Data.}

\cvskill{LEADERSHIP}{Empathetic leadership and coaching, agile/lean project leadership, change management.}

\cvskill{ARCHITECTURE}{Distributed architectures, modular monoliths (clean/hexagonal architecture), HTTP API interface design.}

\cvskill{DOMAIN EXPERTISE}{Data integration, governance, interoperability, standardization, cataloguing, domain modeling, robotics/AI.}

\cvskill{SOFT SKILLS}{Critical and conceptual thinking, collaboration/teamwork, clear communication of complex concepts.}

%\cvskill{OTHER}{}


\end{cvskills}

%-------------------------------------------------------------------------------
%	SECTION TITLE
%-------------------------------------------------------------------------------
\cvsection{Education}


%-------------------------------------------------------------------------------
%	CONTENT
%-------------------------------------------------------------------------------
\begin{cventries}

%---------------------------------------------------------
  \cventry
    {Master of Science (M.Sc), Computer Science} % Degree
    {University of Central Florida} % Institution
    {Orlando, FL} % Location
    {Aug. 2008 - May. 2011} % Date(s)
    {
    }

  \cventry
    {Bachelor of Science (B.Sc.), Computer Science} % Degree
    {} % Institution
    {} % Location
    {Aug. 2001 - May. 2006} % Date(s)
    {
	\begin{cvitems}
		\item {Recipient of Florida Bright Futures Scholarship}
	\end{cvitems}
    }

%---------------------------------------------------------
\end{cventries}

%-------------------------------------------------------------------------------
%	SECTION TITLE
%-------------------------------------------------------------------------------
\cvsection{Publications}


%-------------------------------------------------------------------------------
%	SUBSECTION TITLE
%-------------------------------------------------------------------------------
%\cvsubsection{Databases}


%-------------------------------------------------------------------------------
%	CONTENT
%-------------------------------------------------------------------------------
\begin{cvskills}


\cvskill{2021}{Data-Sharing Markets for Integrating IoT Data Processing Functionalities. \emph{CCF Transactions on Pervasive Computing \&  Interaction}}

\cvskill{2018}{Semantic Data Management for Experimental Manufacturing Technologies. \emph{Datenbank-Spektrum}}

\cvskill{2010}{Focused Belief Revision as a Model of Fallible Relevance-Sensitive Perception. \emph{K\"{u}nstliche Intelligenz}}

\cvskill{2010}{High-Level Perception as Focused Belief Revision. \emph{European Conference on AI (ECAI)}}



\end{cvskills}


%-------------------------------------------------------------------------------
%	SECTION TITLE
%-------------------------------------------------------------------------------
\cvsection{Languages}


%-------------------------------------------------------------------------------
%	SUBSECTION TITLE
%-------------------------------------------------------------------------------
%\cvsubsection{Databases}


%-------------------------------------------------------------------------------
%	CONTENT
%-------------------------------------------------------------------------------
\begin{cvskills}


\cvskill{ENGLISH}{Fluent}

\cvskill{ARABIC}{Fluent}

\cvskill{GERMAN}{B1 Telc Certified (Good Understanding, Basic Speaking)}


\end{cvskills}




%-------------------------------------------------------------------------------
\end{document}
